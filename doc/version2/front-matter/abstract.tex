\chapter{Abstract}
  With the price of wireless sensor technologies diminishing rapidly we can
  expect large numbers of autonomous sensor networks being deployed in the near
  future. These sensor networks will typically not remain isolated but the need
  of interconnecting them on the network level to enable integrated data
  processing will arise, thus realizing the vision of a global ``Sensor
  Internet.'' This requires a flexible middleware layer which abstracts from
  the underlying, heterogeneous sensor network technologies and supports fast
  and simple deployment and addition of new platforms, facilitates efficient
  distributed query processing and combination of sensor data, provides support
  for sensor mobility, and enables the dynamic adaption of the system
  configuration during runtime with minimal (zero-programming) effort. This
  paper describes the Global Sensor Networks (GSN) middleware which addresses
  these goals. We present GSN's conceptual model, abstractions, and
  architecture, and demonstrate the efficiency of the implementation through
  experiments with typical high-load application profiles. The GSN
  implementation is available from \url{http://gsn.sourceforge.net/}.	
  
  
  
\begin{table*}[!htp]
	\centering
	{\normalfont\footnotesize
	\begin{tabulary}{\textwidth}{|C|}%
	\hline
	\\
		\textbf This documentation is a work in progress \\
		\\
		If you find anything here to be incorrect, misleading or incomprehensible please contact us via the GSN Developers list at gsn-devel@lists.sourceforge.net.  \\
		\\
	\hline

	\end{tabulary}
	}
\end{table*}



 
 \newpage