\chapter{Virtual Sensors}
\section{Introduction}
Virtual Sensor is the main abstraction used by GSN to represent a well structured data stream.
By well structured we mean, the structure of the stream is known in advanced and it is not going to be changed
while GSN is running.
\section{Graphical Representation}
\section{}

\section{Notes}
\subsection{What if the structure of the virtual sensor changes?}
We are not changing the structure of the database automatically (adding or droping fields). If your virtual sensors structure
changes and if you are using \emph{permannet storages} such as MySQL or File-based HSQLDB, you have to change the structure of the
table manually. 
\subsection{Where is the DTD for the virtual sensor?}
We are using JiBX Java-XML binding project. The structure of the virtual sensor descriptor file is defined in \texttt{conf/VirtualSensorDescription.xml}.
\subsection{Which databases we support ?}
At the moment we are supporting HSqlDB and MySQL. Checkout the mailing list for the latest issues regarding the other databases and their
support status.
\subsection{Which projects are using GSN ?}
There are over 10 EU/Swiss funding research projects using GSN as their core technology.
\section{GSN.XML file}
\section{SafeStorage}
\subsection{Why we need it? }
\subsection{How to use it }
\subsection{How to write a wrapper for the safe storage}
\section{Wrappers}
\section{Their mandatory parameters}
\section{How it works}
\section{Introduction}
\section{GeoRSS}
\section{SQL Syntax}
\section{Introduction}
\section{Introduction}
\section{Introduction}
\section{Introduction}
\section{Introduction}
\section{Introduction}
\section{Introduction}
\section{Network communication}
Looking inside the GSN infrastructure, there are at least half a dozen difference network communication channels are used. In this section I would like to dive in to the details
of the some major communication protocols designed and implemented in GSN. 

\subsection{Reusing Data Streams}

One of the main ideas behind the virtual sensors is resuability. The resuability comes in two forms.
First being able to recreate the same processing logic on different data streams.
Second being able to reuse streaming data produced by other parties over the internet and possibly create a new data stream but instrumenting the original streams.
In this section, I present the both high level and low level details associated with the second aspect of the reusability.

The virtual sensor descriptor file is the first place which specifies the intention of reusing streaming data from another virtual sensor. The source virtual sensor can be located
anywhere as long as it is accessible through the network, this ofcourse includes the local machine and any other machine on the Internet. 


In GSN, our vision is having an internet scale streaming world in which people can publish streaming data which
can be produced directly using some sort of a measurement device which can range from a physical wireless sensor to stock ticks from a financial market. 
