\graphicspath{{chapters/ch-developer-guide/figures/}}

\chapter{Developer's Guide}

\section{How to develop a wrapper}

\subsection{How to develop a Standard Wrapper  \label{sec:developer-guide_wrappers}}

\subsection{How to develop a Safe Storage Wrapper}

This section describes step by step the development of wrappers that support the Safe Storage feature described in the \chapref{safe_storage}.
As an example to picture this development we use the Safe Storage Memory Monitor wrapper (\inlinecode{ss\_mem\_wrapper}), based on the 
standard one (\inlinecode{memory-usage}).

\begin{enumerate}
	\item In the package \inlinecode{gsn.acquisition2.wrappers},
        create the class that will execute in the Safe Storage process. By convention we name these classes like the following:
        \inlinecode{<Old wrapper name>2.java} (for our example \inlinecode{MemoryMontoringWrapper2.java}). This class must extend the abstract class
        \inlinecode{gsn.acquisition2.wrappers.AbstractWrapper2}.

	\item Add a short name link for this new class in the file:\\ \inlinecode{conf/safe\_storage\_wrappers.properties}.\\For our example we added: \\
        \inlinecode{mem2=gsn.acquisition2.wrappers.MemoryMontoringWrapper2}.

	\item Create the class that will execute on the GSN process. By convention we name these classes like the following:
	\inlinecode{<Old wrapper name>Processor.java} (for our example \inlinecode{MemoryWrapperProcessor}). This class must extend the abstract class
	\inlinecode{gsn.acquisition2.wrappers.SafeStorageAbstractWrapper}.

	\item Add a short name definition to this new class in the file:\\ \inlinecode{conf/wrappers.properties}.\\ For our example we added:\\
	\inlinecode{ss\_mem\_processor=gsn.acquisition2.wrappers.MemoryWrapperProcessor}.

	\item Create a Virtual Sensor Description file for the sensor you will use to test your wrapper. For our example we created the file:\\
	\inlinecode{virtual-sensors/safe-storage/ss\_mem\_vs.xml}. \\
	Your \vsd file must contain at least the following predicates which are mandatory for Safe Storage feature as shown in the \tableref{table:safe_storage_parameters}.
	\begin{xmlcode}
	<address wrapper="GSN_SHORT_NAME (eg. ss_mem_processor)">
		<predicate key="ss-host">SAFE_STORAGE_HOST (default: localhost)</predicate>
		<predicate key="ss-port">SAFE_STORAGE_PORT (default: 25000)</predicate>
		<predicate key="wrapper-name">SAFE_STORAGE_SHORT_NAME (eg. mem2)</predicate>
	</address>
	\end{xmlcode}

	\item Edit the wrapper class that runs on the Safe Storage process (for our example: \inlinecode{MemoryMontoringWrapper2}).
	Four methods have to be implemented

	\begin{itemize}
		\item \inlinecode{boolean initialize ()} \\
		This method is called after the instanciation of the wrapper class.
		It is used to create the resources according to the parameters set in the configuration file. 
		Your implementation must return false if some mandatory parameters were missing or if any error
		arise during this phase. 
		In the other case, this method must return true.
		%You can get the parameters values from the \vsd file (tags \inlinecode{<predicate key="..../>}) with the 
		%\inlinecode{getPredicate[s|Value|ValueAsInt|...]} methods on the ActiveAddressBean instance accessible with the super class 
		%\inlinecode{getActiveAddressBean()} method. Some of these accessors
		%throw (Unmanaged) \inlinecode{RuntimeException} if the parameter if missing and should be used for mandatory parameters. If you have parameters that are
		%optionnal, you should use the \inlinecode{getPredicateValueWithDefault(key, defaultValue)} accessor.
		
		\item \inlinecode{void finalize ()} \\
		This method should be used to free the wrapper ressources. It is called when the wrapper is unloaded.

		\item \inlinecode{String getWrapperName ()} \\
		This method returns the SafeStorage wrapper's name.

		\item \inlinecode{void run ()} \\
		Each wrapper runs in a separate thread. Use the run method to get data from your device and store them to
		the Safe Storage database with the method \inlinecode{postStreamElement (Serializable[])}.
		Notice that you can pass has many parameters has you want to this method but you must set the current time as the last parameter. 
		This parameter will tag the creation of the packet in SafeStorage and is only necessary for running SafeStorage. %TODO remove this parameter in the code.

		For our example: \\
		\inlinecode{postStreamElement(heapMemoryUsage,nonHeapMemoryUsage,pendingFinalizationCount,System.currentTimeMillis());}
	\end{itemize}

	\item Edit the wrapper class that runs on the GSN part (in our case \inlinecode{MemoryWrapperProcessor}).
	Two methods have to be implemented

	\begin{itemize}
		\item \inlinecode{DataField[] getOutputFormat ()} \\
		This method return an array of fields (names, and types) that are produced by the \wrapper.

		\item \inlinecode{boolean messageToBeProcessed (DataMsg dataMessage)} \\
		This method is called upon reception of a new message from the SafeStorage.
		You can access the data (an array of Serializable objects) with the instance method \inlinecode{dataMessage.getData()}.
		The last element in this array is still the TimeStamp that you added before.\\ % TODO remove this parameter in the code
		Once you have parsed your data, you must use one of the \inlinecode{postStreamElement()} method to store data into GSN.
%
%		\item \inlinecode{postStreamElement(long timestamp, Serializable[] values)} \\
%		use the specified timestamp.
%	
%		\item \inlinecode{postStreamElement(Serializable[] values)} \\
%		use the current time as timestamp.
	\end{itemize}

	\item If you need to get some parameters from the VS XML configuration file, to initialize or finalize your wrapper, you can override 
	the superclass methods:

	\begin{itemize}
		\item \inlinecode{boolean initialize()}
		\item \inlinecode{void finalize()}
	\end{itemize}
\end{enumerate}

\section{How to cutomize the GSN Reports}