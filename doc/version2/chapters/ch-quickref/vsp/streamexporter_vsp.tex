\subsection{gsn.vsensor.StreamExporterVirtualSensor \vsp \label{streamexporter:vsp}}

This virtual sensor exports the data it receives to the database of your choice.
This can be interesting when debugging your GSN configuration or to easily
back up critical data on an independent machine. It can also be used to log
unexpected events for later off-line, manual analysis. It can receive any number
of input streams. Each one will be saved into a separate table named after the
input stream.

It requires a JDBC URL and a user name and password so that it knows
where is the database server and how to authenticate. This virtual sensor inserts
the data into the specified table. If the table doesn't exist, GSN first create ths
table and if the table exists, GSN first verifies the structure of the table with
the structure of the produced stream elements. If the structure matches, GSN
starts inserting data into the table otherwise it stops with an error message to
inform the user about the uncompatible strucutres.

\begin{table*}[!htp]
	\centering
	{\normalfont\footnotesize
	\begin{tabulary}{\textwidth}{|C|C|C|C|J|}%
	\hline
		\multicolumn{5}{|c|}{\textbf{Parameters for gsn.vsensor.StreamExportVirtualSensor \vsp}} \\
	\hline
	\hline
		\textbf{Parameter Name} &
		\textbf{Type} &
		\textbf{Mandatory} &
		\textbf{Default} &
		\textbf{Description} \\
	\hline
	\hline
		url &
		String &	
		Yes &
		None &
		A JDBC url that specifies how to connect to the database server. \\
	\hline
		user &
		String &	
		Yes &
		None &	
		The username for authentication with the database server. \\
	\hline
		password &
		String &	
		Yes &
		None &	
		the password for authentication with the database server. \\
	\hline
		driver &
		String &	
		Yes &
		None &	
		The name of the database driver. \\
	\hline
		table &
		String &	
		Yes &
		None &	
		The name of the table into which GSN will store the sensor data. \\
	\hline
	\end{tabulary}
	}
	\caption{Parameters for gsn.vsensor.EmailVirtualSensor \vsp}
	\label{table:parameters_email_vsp}
\end{table*}