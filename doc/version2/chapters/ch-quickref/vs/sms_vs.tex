\subsection{SMS Notification \vs \label{sms:vsp}}

This virtual sensor implements SMS (Short Message Service) notification. This virtual sensor is very similar to the email virtual sensor – an email is sent to a Mobile phone operator or SMS gateway provider based on the user's mobile account and the email is converted and send as a SMS to the given phone number.
This \vs is based on the \chapref{sms:vsp} \vsp and the \chapref{multiformat:wrapper} \wrapper.
The \listingref{listing:xml:smsnotification_vsd} shows an example of \vsd for this \vs.

\begin{xmlcode}[caption={Sample of SMS Notification VSD file}, label=listing:xml:smsnotification_vsd]
<virtual-sensor name="sms" priority="10">
	<processing-class>
		<class-name>gsn.vsensor.SMSVirtualSensor</class-name>
		<init-params>
			<param name="phone-number">004413243545</param>
			<param name="password">3524</param>
			<param name="sms-server">vodafone.co.uk</param>
			<param name="message-format">Temperature: $TEMP$</param>
		</init-params>
		<output-structure>
			<field name="temp" type="double" />
		</output-structure>
	</processing-class>
	<description>Send a SMS Notification</description>
	<life-cycle pool-size="10" />
	<addressing />
	<storage history-size="10m" />
	<streams>
		<stream name="in1">
			<source alias="s1" sampling-rate="1" storage-size="1">
				<address wrapper="multiformat">
					<predicate key="HOST">localhost</predicate>
					<predicate key="PORT">22001</predicate>
				</address>
				<query>SELECT * FROM wrapper</query>
			</source>
			<query>SELECT temperature FROM s1 WHERE temperature >= 100</query>
		</stream>
	</streams>
</virtual-sensor>
\end{xmlcode}

Note that this virtual sensor will only work if you have an account with a mobile phone operator or an internet SMS gateway provider. 