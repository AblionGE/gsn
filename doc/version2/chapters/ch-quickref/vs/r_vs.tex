\subsection{R \vs \label{r:vsp}}
The software architecture of GSN allows the integration of third-party numerical packages such as Matlab, Mathematica, and more recently the R statistical package. In this section, we describe how to integrate and perform data analysis using R and GSN. 

\subsubsection{Installation and Setup}
R is a programming language and a software package for statistical computing and analysis. R provides a range of techniques for statistical analysis such as liner and nonlinear modeling, classical statistical test, time-series analysis. The R software platform is composed of an engine that can interpret and execute R scripts or programs (in a similar way to GNU Octave or Matlab). The R engine can be invoked from the console or over the network using a TCP/IP server called Rserve. In this section, we describe how to configure the R and the Rserve server.

\subsubsection*{Installation}
First, we have to install R. R is available in several platforms, depending on your platform you will have to follow specific instructions. We recommend to read the documentation from the official project website: \url{http://www.r-project.org/}. 

\subsubsection*{Rserve TCP/IP Server}
Rserve is a TCP/IP server which allows other programs to use invoke R from various languages without the need to initialize R or link against R library. Rserve can be downloaded from the following website: \url{http://www.rforge.net/Rserve/}. Rserve is written in Java and it has bindings for other languages.
Rserve comes as a R package, therefore, to install and run Rserve, the library and the package have to be invoked within R as follows:

\begin{bashcode}[caption={Running R and starting Rserve server.}, label=listing:bash:r_vsd]
user@host\# R
R version 2.6.2 (2008-02-08) 
Copyright (C) 2008 The R Foundation for Statistical Computing 

> library(Rserve) 
> Rserve()
\end{bashcode}

This will load the Rserve library and start the Rserve TCP/IP server on the localhost. The default port is 6311. To read more about how to invoke R using Rserve, please refer to the Rserve documentation website: \url{http://www.rforge.net/Rserve/}.

This \vs is based on the \chapref{voip:vsp} \vsp and the \chapref{multiformat:wrapper} \wrapper.
The \listingref{listing:xml:r_vsd} shows an example of \vsd for this \vs.

\begin{xmlcode}[caption={Sample of R VSD file}, label=listing:xml:voip_vsd]
<virtual-sensor name="plot" priority="10">
	<processing-class>
	<class-name>gsn.vsensor.RVirtualSensor</class-name>
		<init-params>
      <param name="plot">2D</param>
			<param name="device">jpeg</param>
			<param name="operation">x<-rnorm(10);plot(x,$temperature$);</param>
		</init-params>
		<output-structure>
			<field name="plot" type="binary:image/jpeg"/>
		</output-structure>
	</processing-class>
	<description>Plots a graph using R.</description>
	<life-cycle pool-size="10"/>
	<addressing>
		<predicate key=""></predicate>
	</addressing>
	<storage history-size="1h"/>
  <streams>
    <stream name="input1">
    <source alias="source1" sampling-rate="1" storage-size="1">
      <address wrapper="multiformat">
          <predicate key="HOST">localhost</predicate>
          <predicate key="PORT">22001</predicate>
      </address>
      <query>SELECT temperature, timed FROM wrapper WHERE packet_type=2</query>
    </source>
    <query>SELECT temperature, timed FROM source1</query>
    </stream> 
  </streams>
</virtual-sensor>
\end{xmlcode}
